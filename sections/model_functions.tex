\textbf{\textcolor{training-set}{Input Populations}}:\\[2pt]
\footnotesize
\textbf{PM}: Encodes a specific coordinate in the reaching space (figure 1). DA establishes connections with active D1 patches, which map the joint angles to that coordinate\\[2pt]
\textbf{S1}: Encodes current joint angles via gain fields \parencite{pougetComputationalApproachesSensorimotor2000}. S1 sends projections that diverge to innervate multiple regions in the Striatum (D1 patches) \parencite{flahertyCorticostriatalTransformationsPrimate1991}\\[2pt]
\textbf{CM} (just in Training): Sends internal state to the GPe and M1. The internal state can be generated by the cerebellum \parencite{sendhilnathanCerebrocerebellarNetworkLearning2024} and forwarded from there to the CM to the BG\\[10pt]
\small
\textbf{Recurrent Populations:}\\[2pt]
\footnotesize
\textbf{D1}: Activity in S1 and M1 activates certain patches in D1 receptors in the striatum\\[2pt]
\textbf{SNr}: SNr is inhibited by the GPe. Active D1 neurons connect with particularly inactive SNr neurons, which leads to increased inhibition\\[2pt]
\textbf{VL}: Inactive SNr neurons disinhibit VL neurons\\[2pt]
\textbf{M1}: M1 encodes shoulder like and elbow like neurons via population code 
\parencite{pruszynskiPrimaryMotorCortex2011}, which map the joint angle to a specific coordinate. The weighted sum of the M1 activity is the angle of the corresponding joint\\[2pt]
\textbf{SNc}: Dopamine is released through movement \parencite{cheungLearningCriticallyDrives2023}, which makes the connections between PM $\rightarrow$ D1 and D1 $\rightarrow$ SNr plastic. Active D1 neurons inhibit DA release (\textbf{RPE})
